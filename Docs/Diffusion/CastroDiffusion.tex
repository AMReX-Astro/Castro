\label{ch:diffusion}

\section{Thermal Diffusion}

Castro incorporates explicit thermal diffusion into the energy equation.  
In terms of the specific internal energy, $e$, this appears as:
\begin{equation}
\rho \frac{De}{Dt} + p \nabla \cdot \ub = \nabla \cdot \kth \nabla T
\end{equation}
where $\kth$ is the thermal conductivity, with units
$\mathrm{erg~cm^{-1}~s^{-1}~K^{-1}}$.

To see the similarity to the thermal diffusion equation, consider the special
case of constant conductivity, $\kth$, and density, and assume an
ideal gas, so $e = c_v T$, where $c_v$ is the specific heat at constant volume.
Finally, ignore hydrodynamics, so $\ub = 0$.  This gives:
\begin{equation}
\frac{\partial T}{\partial t} = D \nabla^2 T
\end{equation}
where $D \equiv \kth/(\rho c_v)$.  Solving this equation
explicitly requires a timestep limiter of
\begin{equation}
\Delta t_\mathrm{diff} \le \frac{1}{2} \frac{\Delta x^2}{D}
\end{equation}
(this is implemented in \code{ca\_estdt\_temp\_diffusion} in {\tt
  Castro/Source/driver/timestep.F90}).

Support for diffusion must be compiled into the code by setting {\tt
  USE\_DIFFUSION = TRUE} in your {\tt GNUmakefile}.  It is treated
explicitly, by constructing the contribution to the evolution as a
source term.  This is time-centered to achieve second-order accuracy
in time.

The following parameter affects diffusion:
\begin{itemize}
\item \runparam{castro.diffuse\_temp}:  enable thermal diffusion (0 or 1; default 0)
\end{itemize}

A pure diffusion problem (with no hydrodynamics) can be run by setting
\begin{verbatim}
castro.diffuse_temp = 1
castro.do_hydro = 0
\end{verbatim}

To complete the setup, a thermal conductivity must be specified.  The
interface for the conductivity is:
\begin{lstlisting}[language=fortran]
  subroutine thermal_conductivity(eos_state, therm_cond)
    
    use extern_probin_module, only: const_conductivity

    type (eos_t), intent(in) :: eos_state
    real (kind=dp_t), intent(inout) :: therm_cond
\end{lstlisting}
The density, temperature, and mass fractions come in through the {\tt
  eos\_state} type.  An EOS call is done in \castro\ just before the
call to \code{thermal\_conductivity}, so you can assume that the entire
state is consistent.

There are two conductivity routines provided with \castro\ by default:
\begin{itemize}
\item {\tt constant} : A simple constant thermal conductivity.  This can be 
  selected by setting 
\begin{verbatim}
Conductivity_dir := constant
\end{verbatim}
in your {\tt GNUmakefile}.  To set the value of the conductivity (e.g., to
$100$), you add to your {\tt probin} file's {\tt \&extern} namelist:
\begin{verbatim}
const_conductivity = 100.0
\end{verbatim}

\item {\tt constant\_opacity} : A simple constant opacity.  This is
  converted to an opacity as:
  \begin{equation}
    \kth = \frac{16 \sigma_B T^3}{3 \kappa_\mathrm{const} \rho}
  \end{equation}
where $\kappa_\mathrm{const}$ is the opacity, with units $\mathrm{cm^2~g^{-1}}$.
This is selected by setting
\begin{verbatim}
Conductivity_dir := constant_opacity
\end{verbatim}
in your {\tt GNUmakefile}.  To set the value of the opacity, e.g., to
0.2 (e.g., for electron scattering), set:
\begin{verbatim}
const_opacity = 0.2
\end{verbatim}
in the {\tt \&extern} namelist of your {\tt probin}.

\end{itemize}

The diffusion approximation breaks down at the surface of stars,
where the density rapidly drops and the mean free path becomes 
large.  In those instances, you should use the flux limited diffusion
module in Castro to evolve a radiation field.  However, if your
interest is only on the diffusion in the interior, you can use
the \runparam{castro.diffuse\_cutoff\_density} parameter to specify a density,
below which, diffusion is not modeled.  This is implemented in the
code by zeroing out the conductivity and skipping the estimation
of the timestep limit in these zones.

A simple test problem that sets up a Gaussian temperature profile 
and does pure diffusion is provided as {\tt diffusion\_test}.



\section{Enthalpy Diffusion}

\castro\ can also diffuse enthalpy \MarginPar{these need to be documented}

Note this uses the same interface for the transport coefficients as
thermal diffusion, so the two cannot be used at the same time.


\section{Species Diffusion}

\castro\ can also diffuse species.  

Note this uses the same interface for the transport coefficients as
thermal diffusion, so the two cannot be used at the same time.



\section{Viscosity}

