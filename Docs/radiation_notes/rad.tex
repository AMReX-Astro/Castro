\documentclass[12pt]{article}

\usepackage[margin=1in]{geometry}

\usepackage{amsmath}
\usepackage{amssymb}

\usepackage{url}

\usepackage{mathpazo}

\newcommand{\nup}{{(\nu)}}
\newcommand{\evm}{{(-)}}
\newcommand{\evz}{{(0)}}
\newcommand{\evp}{{(+)}}

\begin{document}

\begin{center}
{\Large Treatment of gas pressure interface state in Castro + radiation}
\end{center}

The interface states for radiation work in the primitive variable
system, $q = (\rho, u, p, (\rho e)_g, E_r)^\intercal$, where $p$ is
the gas pressure only, and $(\rho e)_g$ is the gas energy density.

Written in the form:
\begin{equation}
q_t + A(q) q_x = 0
\end{equation}
The matrix $A$ takes the form:
\begin{equation}
A = \left (
\begin{matrix}
u & \rho & 0 & 0 & 0\\
0 & u & {1}/{\rho} & 0 & \lambda_f/{\rho}\\
0 & c_{g}^{2} \rho & u & 0 & 0\\
0 & h_{g} \rho & 0 & u & 0\\
0 & E_{r} \left(\lambda_f + 1\right) & 0 & 0 & u
\end{matrix}\right )
\end{equation}
In constructing the interface states, start with a reference state,
$q_\mathrm{ref}$ and define the jumps carried by each wave as the
integral under the parabolic profile with respect to this reference
state:
\begin{equation}
\Delta q^\nup \equiv q_\mathrm{ref} - \mathcal{I}^\nup(q)
\end{equation}
and then the interface states are:
\begin{equation}
q_\mathrm{int} = q_\mathrm{ref} - \sum_\nu (l^\nup \cdot \Delta q^\nup) r^\nup
\end{equation}
Defining:
\begin{equation}
\beta^\evm = \frac{1}{2 c^{2}} \left(
    \Delta E_r^\evm \lambda_f + \Delta p^\evm - \Delta u^\evm c \rho\right)
\end{equation}
in Castro, we write this as:
\begin{equation}
\beta^\evm = \frac{1}{2 c^{2}} \left(
    \Delta p_\mathrm{tot}^\evm - \Delta u^\evm c \rho\right)
\end{equation}
recognizing that $p_\mathrm{tot} = p + \lambda_f E_r$.
Similarly, we have:
\begin{equation}
\beta^\evp = \frac{1}{2 c^{2}} \left(
    \Delta E_r^\evp \lambda_f + \Delta p^\evp + \Delta u^\evp c \rho\right)
\end{equation}
and for the 0-wave, we have:
\begin{align}
\beta^\evz_\rho &= 
    \Delta\rho^\evz  - \frac{\Delta p^\evz_\mathrm{tot}}{c^2} \\
%
\beta^\evz_{{\rho e}_g} &= \Delta(\rho e)^\evz_g - \frac{\Delta p_\mathrm{tot}^\evz}{c^2} h_g \\
%
\beta^\evz_{E_r} &= \Delta E_r^\evz - \frac{\Delta p_\mathrm{tot}^\evz}{c^2} h_r
\end{align}
where $h_r = (\lambda_f + 1)E_r/\rho$.  Note, these match the derivation
done in the Jupyter/SymPy notebook:\newline
{\footnotesize \url{https://github.com/zingale/hydro_examples/blob/master/compressible/euler-generaleos.ipynb}}

The gas pressure update in these terms is:
\begin{equation}
p_\mathrm{int} = p_\mathrm{ref} - (\beta^\evp + \beta^\evm) c_g^2 + \lambda_f \beta^\evz_{E_r}
\end{equation}
This matches the expression in Castro.  But the question is---what do
we do about the reference state?  The quantity $\Delta
p^\evz_\mathrm{tot}$ is defined as:
\begin{equation}
\Delta p^\evz_\mathrm{tot} = 
  p_\mathrm{tot,ref} - \mathcal{I}^\evz(p_\mathrm{tot})
\end{equation}

Note the potential inconsistency here.  We want the expression for
$p_\mathrm{int}$ to start with $p_\mathrm{ref}$, since, if no waves
are moving toward the interfaces, the $\beta$'s are $0$ and we have
$p_\mathrm{int} = p_\mathrm{ref}$.

But in the case that all the waves are moving toward the interface, then
we are projecting against $p_\mathrm{tot,ref}$.  This is inconsisent, since
if we were linear, then the reference state should simply cancel out.

\end{document}
