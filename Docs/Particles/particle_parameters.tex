\section{ {\tt castro } Namespace}

\label{ch:parameters}


%%%%%%%%%%%%%%%%
% symbol table
%%%%%%%%%%%%%%%%

\begin{landscape}



{\small
	
	\renewcommand{\arraystretch}{1.5}
	%
	\begin{center}
		\begin{longtable}{|l|p{5.25in}|l|}
			\caption[castro :  AMR
			parameters]{castro :  AMR
				parameters} \label{table: castro :  AMR
				parameters runtime} \\
			%
			\hline \multicolumn{1}{|c|}{\textbf{parameter}} & 
			\multicolumn{1}{ c|}{\textbf{description}} & 
			\multicolumn{1}{ c|}{\textbf{default value}} \\ \hline 
			\endfirsthead
			
			\multicolumn{3}{c}%
			{{\tablename\ \thetable{}---continued}} \\
			\hline \multicolumn{1}{|c|}{\textbf{parameter}} & 
			\multicolumn{1}{ c|}{\textbf{description}} & 
			\multicolumn{1}{ c|}{\textbf{default value}} \\ \hline 
			\endhead
			
			\multicolumn{3}{|r|}{{\em continued on next page}} \\ \hline
			\endfoot
			
			\hline 
			\endlastfoot
			
			
			\rowcolor{tableShade}
			\runparamNS{particles.particle\_init\_file}{castro} &  This indicates the name of an input file containing the total particle number and the initial position of each particle. For example, if {\tt particles.particle\_init\_file}={\em particle\_file}, the initial data of particles should be stored in a file named {\em particle\_file}. &  \\
			\runparamNS{particles.particle\_restart\_file}{castro} &  We can add new particles at restart and this variable refers to the name of a file, containing the information of a new set of particles, in the the same format as the input file from {\tt particles.particle\_init\_file} &  \\
			\rowcolor{tableShade}
			\runparamNS{particles.timestamp\_dir}{castro} &  This variable defines the name of a directory for output. The directory is made as a simulation starts and the output files are stored in the directory. &  \\
			\rowcolor{tableShade}
			\runparamNS{particles.timestamp\_density}{castro} &  This decides whether the local densities at given positions of particles are stored (stored if 1)  &  0\\
			\runparamNS{particles.timestamp\_temperature}{castro} &  This decides whether the local temperature at given positions of particles are stored (stored if 1) & 0 \\
						\rowcolor{tableShade}
\runparamNS{particles.v }{castro} &  verbosity  &  0\\
\runparamNS{particles.write\_in\_plotfile }{castro} &  The particle positions and velocities will be written in a binary file in each plotfile directory if 1. & 1 \\			
		\end{longtable}
	\end{center}
	
} % ends \small




\end{landscape}

%


