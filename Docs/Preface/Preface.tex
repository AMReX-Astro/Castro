
Welcome to the \castro\ User's Guide!

In this User's Guide we describe how to download and run \castro, a
massively parallel code that solves the multicomponent compressible
hydrodynamic equations for astrophysical flows including self-gravity,
nuclear reactions and radiation.  \castro\ uses an Eulerian grid and
incorporates adaptive mesh refinement (AMR).  Our approach to AMR uses
a nested hierarchy of logically-rectangular grids with simultaneous
refinement in both space and time, utilizing the
\amrex\ library\footnote{earlier versions of \castro\ used the
  \boxlib\ library}.

The core algorithms in \castro\ are described in a series of papers:
\begin{itemize}
\item {\it CASTRO: A New Compressible Astrophysical Solver. I. Hydrodynamics and Self-gravity},
  A.~S.~Almgren, V.~E.~Beckner, J.~B.~Bell, M.~S.~Day, L.~H.~Howell, C.~C.~Joggerst, M.~J.~Lijewski,
  A.~Nonaka, M.~Singer, \& M.~Zingale, 2010, ApJ, 715, 1221\newline
  \url{http://dx.doi.org/10.1088/0004-637X/715/2/1221}

\item {\it CASTRO: A New Compressible Astrophysical Solver. II. Gray Radiation Hydrodynamics},
  W.~Zhang, L.~Howell, A.~Almgren, A.~Burrows, \& J.~Bell, 2011, ApJS, 196, 20\newline
  \url{http://dx.doi.org/10.1088/0067-0049/196/2/20}

\item {\it CASTRO: A New Compressible Astrophysical Solver. III. Multigroup Radiation Hydrodynamics},
  W.~Zhang, L.~Howell, A.~Almgren, A.~Burrows, J.~Dolence, \& J.~Bell, 2013, ApJS, 204, 7\newline
  \url{http://dx.doi.org/10.1088/0067-0049/204/1/7}

\end{itemize}

Improvements to the gravity solver and rotation were described in:
\begin{itemize}
\item {\it Double White Dwarf Mergers on Adaptive Meshes I. Methodology
       and Code Verification, }
  M.~P.~Katz, M.~Zingale, A.~C.~Calder, F.~D.~Swesty, A.~S.~Almgren, W.~Zhang
  2016, ApJ, 819, 94.\newline
  \url{http://dx.doi.org/10.3847/0004-637X/819/2/94}
\end{itemize}

The development of \amrex\ library is led by the
Center for Computational Sciences and Engineering / Lawrence Berkeley
National Laboratory.  \castro\ development is done collaboratively,
including the CCSE and Stony Brook University.

\castro\ {\em core developers} are those who have made substantial
contributions to the code.  The process for becoming a core developer
is described in the {\tt README.md} in the \castro\ root directory.
Current \castro\ core developers are:

% git shortlog -sn
\begin{quote}
Ann Almgren\newline
Maria G.\ Barrios Sazo\newline
John Bell\newline
Vince Beckner\newline
Marc Day\newline
Max Katz\newline
Mike Lijewski\newline
Chris Malone\newline
Andy Nonaka\newline
Weiqun Zhang\newline
Michael Zingale
\end{quote}

All \castro\ development takes place on the project's github
page\\[0.5em] \url{https://github.com/AMReX-Astro/Castro}\\[0.5em]
External contributions are welcomed.  Fork the \castro\ repo, modify
your local copy, and issue a pull-request to the {\tt
  AMReX-Astro/Castro} project.  Further guidelines are given in the
{\tt README.md} file.

To get help, subscribe to the {\em castro-help} google group mailing list:
\url{https://groups.google.com/forum/#!forum/castro-help}


\section*{Acknowledging and Citing \castro}

If you use \castro\ in your research, we would appreciate it if you
cited the relevant code papers describing its design, features, and
testing.  A list of these can be found in the
\href{https://github.com/AMReX-Astro/Castro/blob/master/CITATION}{\tt
  CITATION} file in the root {\tt Castro/} directory.

The development \castro\ is supported by the science application
interests of the contributors.  There is a lot of effort behind the
scenes: testing, optimization, development of new features, bug
fixing, $\ldots$, that is often done under the radar.  Nevertheless,
we are happy to volunteer our time to help new users come up to speed
with \castro.  When significant new development / debugging for you
application is provided by a member of the \castro\ development
community, we would appreciate consideration of inviting the
developer(s) for co-authorship on any science paper that results.
