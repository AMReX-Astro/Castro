
\section{Getting Started With Git and CVS}

CASTRO is now distributed in two parts -- you must download the BoxLib repository and the CASTRO module 
using git.  If you would like to use CASTRO:

\begin{enumerate}

\item Assuming git is installed on your machine -- and we recommend version 1.7.x or higher -- download the
BoxLib repository by typing 

\noindent git clone https://ccse.lbl.gov/pub/Downloads/BoxLib.git 

\noindent This will create a directory called BoxLib on your machine.  Put this somewhere out of the way and 
set the environment variable, {\bf BOXLIB\_HOME}, on your machine to the path name where
you have put BoxLib.    You will want to periodically update BoxLib by typing 

\noindent git pull

in the BoxLib directory.  

\item Once you have an account on battra.lbl.gov (if you don't, then contact Ann Almgren at asalmgren@lbl.gov),
please log on immediately to battra and change your password.
This is the only time you will log in to battra directly.  However, you will use this password every
time you use CVS or git to check out or update code from battra.

\item In order to use CVS to download the CASTRO module, set the following environment variables on your machine: \\ \\ 
CVS\_RSH=ssh \\
CVSROOT=":ext:$username$@battra.lbl.gov:/home/ccse\_src/cvsroot" 

\item Now, you are ready to download the CASTRO module.  In a directory of your choosing, type 

"cvs co CASTRO"

"co" here is a shorthand for "checkout"

"CASTRO" is the name of a CVS module, which we have defined to include all the directories and subdirectories 
that you will need (in addition to BoxLib) to successfully build and run CASTRO.

You will again need to enter your battra password here.

\item Another possibility, if you don't want to set CVS\_RSH and CVSROOT, is to simply type

"cvs -d :ext:$username$@battra.lbl.gov:/home/ccse\_src/cvsroot co CASTRO"

\end{enumerate}

\clearpage

\section{Building the Code}

\begin{enumerate}

\item From the directory in which you checked out CASTRO, type

cd CASTRO/Parallel/Castro/Exec/Sedov

This will put you into a "run" directory in which you can run the Sedov problem in 1-d, 2-d or 3-d.
\item In Sedov, edit the GNUmakefile, and set

DIM = 2 (for example)

COMP = your favorite C++ compiler

FCOMP = your favorite Fortran compiler (which must compile F90)

DEBUG = FALSE

We like COMP = Intel and FCOMP = Intel. Some users successfully use 
COMP = gcc and FCOMP = gfortran. If you want to try other compilers and they don't work, 
please let us know.  (Note from Andy: g++ version 4.4 didn't work with DIM=1.  I had to revert to version 4.1.  gfortran version 4.4 works fine)

To build a serial (single-processor) code, set {\tt USE\_MPI = FALSE}.
This will compile the code without the MPI library.  If you want to do
a parallel run, then you would set {\tt USE\_MPI = TRUE}.  In this
case, the build system will need to know about your MPI installation.
This can be done by editing the makefiles in the BoxLib tree, but a
simple method is to set the shell environment variable {\tt
  BOXLIB\_USE\_MPI\_WRAPPERS=1}.  If this is set, then the build
system will fall back to using the local MPI compiler wrappers
(e.g.\ mpic++ and mpif90) to do the build.

\item Now type "make". The resulting executable will look something like 
"Castro2d.Linux.Intel.Intel.ex", which means this is a 2-d version of the code, 
made on a Linux machine, with COMP = Intel and FCOMP = Intel.

\end{enumerate}

\section{Running the Code}

\begin{enumerate}

\item Type "Castro2d.Linux.Intel.Intel.ex inputs.2d.cyl\_in\_cartcoords" 
This will run the 2-d cylindrical Sedov problem in Cartesian (x-y coordinates). 
You can see other possible options, which should be clear by the names of the inputs files.

\item You will notice that running the code generates directories that look like 
plt00000, plt00020, etc, and chk00000, chk00020, etc. These are "plotfiles" and 
"checkpoint" files. The plotfiles are used for visualization, the checkpoint files are 
used for restarting the code.

\end{enumerate}

\section{Visualization of the Results}

\begin{enumerate}

\item To visualize the plotfiles, you will need to go into the CASTRO/Parallel/pAmrvis 
directory. Edit the GNUmakefile there to set DIM = 2, and again set COMP and FCOMP to 
compilers that you have. Leave DEBUG = FALSE. Then type "make".  This will make an 
executable that looks like "amrvis2d...ex".

[Note from Andy] If you want to build amrvis with DIM = 3, you must first build volpack by 
typing ``make'' in CASTRO/Parallel/volpack.

[Note from Ann] This requires the OSF/Motif libraries and headers, if you don't have these 
you will need to install the development version of motif through your package manager. 
lesstif gives some functionality and will allow you to build the amrvis executable, 
but amrvis will not run properly.

[Note from Shaw] On most Linux distributions, motif library is provided by the openmotif package, 
and its header files (like Xm.h) are provided by openmotif-devel. If those packages are not 
installed, then use the package management tool to install them, which varies from distribution 
to distribution, but is straightforward. I can provide detailed instructions if anyone needs them.

You may then want to create an alias to amrvis2d, for example

alias amrvis2d /work/CASTRO/Parallel/pAmrvis/amrvis2d...ex

\item Return to the CASTRO/Parallel/Castro/Exec/Sedov directory. 
Type "amrvis2d plt00152" to see a single plotfile, or "amrvis2d -a plt*", 
which will animate the sequence of plotfiles. Try playing around with this -- note you can change which 
variable you are looking at, you can select a region and click "Dataset" (under View) in order to look at 
the actual numbers, etc. You can also export the pictures in several different formats -- under "File", see "Export".

Please know that we do have a number of conversion routines to other formats (such as matlab), 
but it is hard to describe them all. If you would like to display the data in another format, 
please let us know (again, asalmgren@lbl.gov) and we will point you to whatever we have that can help.

\end{enumerate}

You have now completed a brief introduction to CASTRO. 
