\section{Code structure}

The code structure in the Castro directory is as follows:

\begin{itemize}
    \item {\bf constants}      : contains a file of useful constants in CGS units
    \item {\bf ConvertCheckpoint} : a tool to convert a checkpoint file to a larger domain
    \item {\bf EOS}            : contains directories for different EOS routines
    \item {\bf Exec}       : various examples
    \begin{itemize}
      \item {\bf Sedov}        : run directory for the Sedov problem
      \item {\bf Sod}          : run directory for the Sod problem
      \item {\bf KH}           : run directory for the Kelvin-Helmholz problem
    \end{itemize}
    \item {\bf Networks}       : contains directories for different reaction networks
    \item {\bf Source}     : source code
    \item {\bf UsersGuide} : you're reading this now!
    \item {\bf Util}       : a catch-all for additional things you may need 
\end{itemize}

\section{Variable Names}

The following is a list of variables, routines, etc used in CASTRO. It may not be complete or even entirely accurate; it's mostly intended for my own use.\\

{\bf lo,hi}: index extent of the "grid" of data currently being handled by a CASTRO routine\\

{\bf domlo, domhi}: index extent of the problem domain. This changes according to refinement level: 0th refinement level will have 0, castro.max\_grid\_size, and nth level will go from 0 to castro.max\_grid\_size*(multiplying equivalent of sum)castro.ref\_ratio(n).\\

{\bf dx}: cell spacing, presumably in cm, since CASTRO uses cgs units\\

{\bf xlo}: physical location of the lower left-hand corner of the "grid" of data currently being handled by a CASTRO routine\\

{\bf bc}: array that holds boundary condition of and array. Sometimes it appears of the form bc(:,:) and sometimes bc(:,:,:). The last index of the latter holds the variable index, i.e. density, pressure, species, etc.\\

{\bf EXT\_DIR}: from  BoxLib/Src/C\_AMRLib/BC\_TYPES.H:EXT\_DIR  : data specified on EDGE (FACE) of bndry\\

{\bf FOEXTRAP}: from  BoxLib/Src/C\_AMRLib/BC\_TYPES.H:FOEXTRAP  : first order extrapolation from last cell in interior CASTRO 

\section{Boundaries}
\subsection{Boundaries Between Grids}
Boundaries between grids are of two types. The first we call "fine-fine", which is two grids at the same level. 
Filling ghost cells at the same level is also part of the fillpatch operation -- it's just a straight copy from "valid regions" 
to ghost cells. The second type is "coarse-fine", which needs interpolation from the coarse grid to fill the fine grid ghost cells. 
This also happens as part of the FillPatch operation, which is why arrays aren't just arrays, they're "State Data", 
which means that the data knows how to interpolate itself (in an anthropomorphical sense). 
The type of interpolation to use is defined in Castro\_setup.cpp as well -- search for cell\_cons\_interp, for example -- 
that's "cell conservative interpolation", i.e the data is cell-based (as opposed to node-based or edge-based) and the 
interpolation is such that the average of the fine values created is equal to the coarse value from which they came. 
(This wouldn't be the case with straight linear interpolation, for example.) 

\subsection{Physical Boundaries}
%%%%%%%%%%%%%%%%%%%%%%%%%%%%%%%%%%%
\begin{table}[h]
\begin{scriptsize}
\begin{center}
\begin{tabular}{|c|c|c|c|} \hline
Physical BC & Velocity & Temperature & Scalars \\
\hline
Outflow & FOEXTRAP & FOEXTRAP & FOEXTRAP \\
No Slip Wall with Adiabatic Temp & EXT\_DIR $u=v=0$ & REFLECT\_EVEN $dT/dt=0$ & HOEXTRAP \\
No Slip Wall with Fixed Temp & EXT\_DIR $u=v=0$ & EXT\_DIR & HOEXTRAP \\
Slip Wall with Adiabatic Temp & EXT\_DIR $u_n=0$, HOEXTRAP $u_t$ & REFLECT\_EVEN $dT/dn=0$ & HOEXTRAP \\
Slip Wall with Fixed Temp & EXT\_DIR $u_n=0$ & EXT\_DIR & HOEXTRAP \\
\hline
\end{tabular}
\end{center}
\caption{Conversions from physical to mathematical BCs}
\label{Table:BC}
\end{scriptsize}
\end{table}
%%%%%%%%%%%%%%%%%%%%%%%%%%%%%%%%%%%
The boundary conditions in Table \ref{Table:BC} have already been implemented in CASTRO.  
The table looks cruddy--it's copied from {\tt BoxLib/Src/C\_AMRLib/amrlib/BC\_TYPES.H}. 
Some of that makes more sense if there are linebreaks within the table, 
but I'm not sure how to do it. Here's definitions of some of the funnier-souding all-caps words from above:\\

INT\_DIR  : data taken from other grids or interpolated

EXT\_DIR  : data specified on EDGE (FACE) of bndry

HOEXTRAP  : higher order extrapolation to EDGE of bndry

FOEXTRAP  : first order extrapolation from last cell in interior

REFLECT\_EVEN : F(-n) = F(n) true reflection from interior cells

REFLECT\_ODD  : F(-n) = -F(n) true reflection from interior cells\\ \\
Basically, boundary conditions are imposed on "state variables" every time that they're "fillpatched", as part of the fillpatch operation.

For example, the loop that calls CA\_UMDRV (all the integration stuff) starts with \\ \\
{\tt       for (FillPatchIterator fpi(*this, S\_new, NUM\_GROW,
                                  time, State\_Type, strtComp, NUM\_STATE);
            fpi.isValid();
            ++fpi)
}\\ \\
Here the FillPatchIterator is the thing that distributes the grids over processors and 
makes parallel "just work". This fills the single patch "fpi" , which has NUM\_GROW 
ghost cells, with data of type "State\_Type" at time "time", starting with component 
strtComp and including a total of NUM\_STATE components.

The way that you tell the code what kind of physical boundary condition to use is given 
in Castro\_setup.cpp. At the top we define arrays such as "scalar\_bc", "norm\_vel\_bc", 
etc, which say which kind of bc to use on which kind of physical boundary. 
Boundary conditions are set in functions like "set\_scalar\_bc", which uses the 
scalar\_bc pre-defined arrays.

If you want to specify a value at a function (like at an inflow boundary), 
there are routines in Prob\_1d.f90, for example, which do that. Which routine is called 
for which variable is again defined in Castro\_setup.cpp 

\section{Parallel I/O}
Both checkpoint files and plotfiles are really directories containing subdirectories: 
one subdirectory for each level of the AMR hierarchy.  The fundamental data structure 
we read/write to disk is a MultiFab, which is made up of multiple FAB's, one FAB per grid.
Multiple MultiFabs may be written to each directory in a checkpoint file.  
MultiFabs of course are shared across CPUs; a single MultiFab may be 
shared across thousands of CPUs.  Each CPU writes the part of the MultiFab that it owns to disk, 
but they don't each write to their own distinct file.  Instead each MultiFab is written to a 
runtime configurable number of files N (N can be set in the inputs file as the
parameter {\bf amr.checkpoint\_nfiles} and {\bf amr.plot\_nfiles}; the default is 64).  
That is to say, each MultiFab is written to disk across at most N files, 
plus a small amount of data that gets written to a header file describing how the 
file is laid out in those N files.

What happens is N CPUs each opens a unique one of the N files into which the MultiFab is 
being written, seeks to the end, and writes their data.  The other CPUs are waiting at a 
barrier for those N writing CPUs to finish.  This repeats for another N CPUs until all the 
data in the MultiFab is written to disk.  All CPUs then pass some data to CPU 0 which writes 
a header file describing how the MultiFab is laid out on disk.

We also read MultiFabs from disk in a "chunky" manner opening only N files for reading at a 
time.  The number N, when the MultiFabs were written, does not have to match the number N when 
the MultiFabs are being read from disk.  Nor does the number of CPUs running while reading in 
the MultiFab need to match the number of CPUs running when the MultiFab was written to disk.

Think of the number N as the number of independent I/O pathways in your underlying parallel 
filesystem.  Of course a "real" parallel filesytem should be able to handle any reasonable 
value of N.  The value -1 forces N to the number of CPUs on which you're running, which means 
that each CPU writes to a unique file, which can create a very large number of files, which 
can lead to inode issues.
